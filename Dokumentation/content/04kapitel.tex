\chapter{Schluss}
\label{sec:schluss}
In diesem letzten Kapitel wird die gesamte Arbeit zusammengefasst und daraufhin eine kritische Reflexion derer durchgeführt. Zum Schluss ist noch ein kleiner Ausblick für die Zukunft gegeben.
\section{Zusammenfassung}
\label{zusammenfassung}
- Was war das Ziel die Motivation\\
    Wir wollen Daten schaffen, die in der Lehre eingesetzt werden können um Verzerrungen von Trainingsdaten zu veranschaulichen.
    Modell lernt aus Daten --> Es lernt auch von Manipulierten Daten oder Verzerrten Daten.
    Trainingdaten können beim Einsatz des Maschinellen Lernen großen Einfluss auf das Ergebnis haben
    Häufig ist der Mensch für Verzerrungen aufgrund von bspw Vorurteilen Verantwortlich
    Vorurteile von Menschen die in Daten auffindbar sind werden sich niemals vermeiden lassen
    Aufgabe ist es Technische Gegenmaßnahme zu finden aber auch die Menschen für das Thema von Bias in Daten zu sensibilisieren
    Es muss mehr Wissen über das noch sehr neue Forschungsgebiet von Bias geschaffen werden, damit im Umgang mit KI darauf geachtet werden kann
- Was sind Daten\\
    Daten haben sich in der Vergangenheit vom Nebenprodukt zu einer Wertvollen Ressource entwickelt
    Durch Big Data und Internet of Things ist das generieren und speichern von großen Daten Mengen kein problem mehr
    Die Qualität von Daten ist häufig entscheidender für die Verwendung
    Qualität ist ein entschiedender Faktor in der Verarbeitung von Daten und lässt sich inzwischen mehr oder weniger anhand von Quaitäts Merkmalen messen
- Was hat es mit Daten un KI auf sich\\
    Daten sind die Grundlage für KI
    Besonders das Teilgebiert superviced Learning aus dem Bereich des ML ist Datengetriebene
    Man lernt Zusammenhänge aus Trainingsdaten und kann ein Modell erzuegen, welches das Verhalten in den historischen Daten zuverlässig reproduzieren kann
    %Hier hast du vllt noch eine Idee wie man da was schreibt sonst passt des 
- Was ist Ethik in der KI \\
    Ethik ist die Grundlage einer Gesellschaft
    Neben dem Recht ist die Ethik eine Leitlinie zum Verhalten in der Gesellschaft 
    Beim Einsatz von KI bekommt die Ethik einen immer größer werdenden stellenwert.
    KI übernimmt immer mehr Entscheidungen, die großen Einfluss auf das Leben von einzelnen Individuen nehmen können
    Um die KI und Ethische Wertvorstellungen zu vereinen, werden leitlinien aus der Wirtschaft und Wissenschaft getrieben.
    Ein Konzept ist die vertrauenswürdige KI - Besthend aus Recht und Ethik. 
    Wenn KI eingesetzt wird, müssen Grundwerte wie: Autonomie/Selbstbestimmtheit des Menschen, fairness, Transparenz, Verlässlichkeit, Sicherheit und Datenschutz gewährleistet werden
    Besonders der Aspekt der Fairness spielt für die Arbeit eine besondere Rolle.
    KI ist in der Lage, abhängig von ihrer Funktionsweise, für unulässige Unfairness zu sorgen. Insbesondere für Entscheidungen spielt Fairness eine sehr große Rolle und muss berücksichtigt werden
- Was ist ein Bias \\
    Bias beschreibt allgemein eine Verzerrung
    Im Kontext von KI ist eine Verzerrung oft eine nicht wahrheitsgemäße repräsentation der Realität, oft durch unzulässige Vorurteile
    
    Im ML gibt es eine vielzahl an unterschiedlichen Arten von Bias
    Bias kann durch Menschen, Daten oder Algorithmen erzeugt werden
    Bei Menschen sind dies in der Regel vorurteile durch die Gesellschaft, das verhalten oder die Vergangenheit geprägt
    Algorithmen können durch falsch erlernte Zusammenhänge falsche Schlüsse ziehen und so Verzerrungen von Ergebnissen erzuegen

    Der entschiedende und für diese Arbeit wichtigste Bias ist jedoch der, der durch die Daten erzeugt wird.
    Es gibt unterschiedliche Arten von Bias in Daten, wie z.B. die Auswahl der Trainingsdaten (sampling Bias) bis hin zu deren korrektheit (measurement Bias)
    
    Eine häufige Art der Verzerrung ist die der Trainingsdaten - Man spricht von Lable Bias 
    Problem ist, die Trainingsdaten werden nahezu immer von Menschen "gelabelt" und so werden im superviced learning Menschliche Verhaltensweisen reproduziert
    Lable Bias ist eigentlich ein durch den Menschen als Bewertenden geschaffenes Problem
    Bei der Erstellung der Lables werden Vorurteile in die Trainingsdaten übertragen 
    %Hier vllt noch ein Beispiel kurz Anreißen, dass KI bspw. Sexistisch Etnscheidet bei Amazon im Einstellungsprozess

%Timo 
- Welche Szenarien\\
- Wie sieht das Ergebnis aus\\
- Wie ist der Bias erkennbar

%Fazit Ziehen  und Kritisch Reflektieren !!!!!
- Der Datengenerator ermöglicht es uns an einem praktischen Beispiel Bias in Daten zu ermitteln
- Man kann das Projekt hinsichtlich des Ziels ein Lehrmittel zu schaffen als vollen Erfolg betrachten
- Man kann Daten generieren, diese individuell verzerren und besitzt eine groß genuge Menge an qualitativ guten Daten für die Verwendung als Trainingsdaten
- Man kann 

Kritische Reflektieren der gesamten Arbeit.\\
%Part Simo
- Was war bei dem Stand der Technik gut schlecht\\
    Es handelt sich um ein erst Junges Forschungsgebiet
    Es gibt beisher wenige Lösungen einen erkannten Bias zu ninimieren oder damit Umzugehen
    Oft ist die einzige Lösung neue Daten zu erzuegen 
    oder einbusen in der Genauigkeit zu bekommen durch das Entfernen von Attributen
    Technisch gibt es viele Herausforderungen.

    Aber auch organisatorisch müssen Methoden entwickelt werden, wie man in der Lage ist frühzeitg Verzerrungen zu erkennen, Gegenmaßnahmen zu definieren und präventiv zu verhindern.

    Man steht in beiden Bereichen noch vor Herausforderungen. Die Auswirkungen von Bias hingegen sind unberechenbar und teils nicht abzusätzen.
    Es kann harmlose Folgen haben, wie einen Skandal, für den man sich entschuldigen muss, aber auch deutlich drastischere Folgen, wenn man sich bspw den Bereich Gesundheitswesen oder Justiz ansieht
    --> Entscheidungen über "leben ond tod" bzw. Einschränkung der Freiheit die lediglich durch einen Bias entstehen könnte.

    EINE KI muss zukünftig vertrausnwürdig sein und das Bedeutet sowohl die rechtlichen Begebenheiten als auch die Gesellschaftlichen ethischen Wertvorstellungen erfüllen in ihrem Handeln.
    Deshalb ist ein erster Schritt die Aufklärung über die existenz von Verzerrungen und Bias!!!
%Part Timo
- Wie war die Umsetzung und Evaluierung\\
- Abschließend wie hat alles in allem geklappt Konnten die Ziele erreicht werden

\section{Ausblick}
\label{ausblick}
- Ausblick wie kann es weiter gehen\\
    KI entwickeln und damit die Daten in ein produktives ML Modell überführen --> noch mehr effekt in der Lehre
    Zukünfitg kann man eine KI entwickeln, diese mit den Trainingsdaten trainieren und den Bias in einem Beispiel veranschaulichen wie es aus der Realität sein könnte
    Technische Lösungen und Ansätze Bias besser zu erkennen, fürhzeitiger zu erkennen 
    Technische Lösungen um Bias zu entfernen --> Viel Forschungsaufwand nötig
    Operationale Lösungen finden, dass menschliche Vorurteile erst gar nicht in Daten repräsentiert werden 
    Konzept für vertrauenswürdige KI --> Zertifizierung von Systemen um sicher zu stellen, dass die Grund anforderungen erfüllt werden können
- Was kann wie gut verwendet werden\\
- Was kann noch verbessert/erweitert werden z.B. Datenauswertung oder neues Szenario oder noch mehr Realitätsnähe\\

- Wo kann weiter geforscht werden\\
    %Word ding auf Teams geschickt, vllt kurz und knapp her als Abschluss rein  Quelle: Fabi (conclusion)
    Dadurch, dass es sich um so einen neuen Forschungsbereich handelt, wurden von dem Thema einige Facetten noch nicht betrachtet
    Eine Veröffentlichung setzt sich mit den vorteilen von Bias im ML auseinandern
    Sie wollen den Bias bewusst in Daten erzeugen um ein Modell für ehtische Richtlinien zu sensibilisieren
    Sie wollen nicht die Sympotem bekämpfen durch das Entfernen von Attributen bspw sondern die Ursache -->x Das Modell selbst
    Anstatt Algorithmen zu manipulieren und Trainingsreize unterdürcken, den Bias bewusst einsetzten
    Die Implementierung ethischer und kognitiver maschineller Verzerrungen kann jedoch dazu beitragen, sich in unsicheren, komplexen und sich schnell verändernden realen Umgebungen effektiv zurechtzufinden, indem sie ein konstantes ethisches Verhalten der Maschinen gewährleistet bzw. die Genauigkeit der algorithmischen Entscheidungsfindung erhöht
    Filtern von Daten ist nur Symptom bekämpfung und nicht die ergründung des problems von Voreingenommenheit egal ob in Lernalgorithmen oder Menschen\cite{Fabi2022}
\newpage