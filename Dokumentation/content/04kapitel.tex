\chapter{Schluss}
\label{sec:schluss}
\begin{onehalfspace}
    In diesem letzten Kapitel wird die gesamte Arbeit zusammengefasst und daraufhin eine kritische Reflexion derer durchgeführt. Zum Schluss ist noch ein kleiner Ausblick für das weitere Vorgehen sowie ein Literaturtipp gegeben.
\section{Zusammenfassung}
\label{zusammenfassung}
Die Motivation und das Ziel der Arbeit lag darin, Daten zu schaffen, welche in der Lehre eingesetzt werden können, um Verzerrungen von Trainingsdaten zu veranschaulichen. Da ein \ac{KI} Modell aus diesen Trainingsdaten lernt und somit lernt solch ein Modell häufig auch von Manipulierten Daten oder Verzerrten Daten. Die Qualität der Trainingsdaten ist hierbei für das Ergebnis des Maschinellen Lernen von sehr großer Bedeutung. So haben solche Verzerrung in den Daten ebenfalls einen sehr großen Einfluss auf das Ergebnis des maschinellen Lernens. In den meisten Fällen ist der Mensch für Verzerrungen aufgrund von beispielsweise Vorurteilen verantwortlich. Diese Vorurteile, welche durch den Menschen in die Daten gelangen, lassen sich niemals vermeiden. Aufgabe ist es stattdessen technische Gegenmaßnahmen zu finden und vor allem den Menschen gegenüber dem Thema Bias in Daten zu sensibilisieren. Daher muss mehr Wissen über das noch sehr neue Forschungsgebiet von Bias geschaffen werden, damit im Umgang mit KI darauf geachtet werden kann.\\
Dies ist die Motivation der Arbeit, doch was sind Daten überhaupt und wie spielt hierbei eine KI und Ethik eine Rolle. Daten haben sich in der Vergangenheit von einem Nebenprodukt zu einer sehr wertvollen Ressource entwickelt. Durch Big Data und Internet of Things ist das Generieren sowie Speichern von großen Daten Mengen kein Problem mehr. Daraus ergeben sich neue Geschäftsbereiche und Entwicklungsmöglichkeiten. Die Qualität der Daten ist dabei häufig der entscheidende Faktor für die Verwendung und lässt sich inzwischen meist anhand von Qualitätsmerkmalen messen. Diese Daten sind die Grundlage für eine KI. Besonders im Teilgebiet des superviced learning aus dem Bereich des ML spielen Daten die Schlüsselrolle zum Erfolg. In diesem lernt man Zusammenhänge aus Trainingsdaten und erzeugt daraus ein Modell, welches das Verhalten in den historischen Daten zuverlässig reproduzieren kann. Dadurch können Prozesse optimiert, automatisiert oder auch ersetzt und einen großen Mehrwert aus den Daten geschaffen werden. Da jedoch aus historischen Daten das Verhalten teilweise von Menschen gelernt wird und die KI immer mehr Entscheidungen übernimmt, welche Einfluss auf das Leben einzelner Individuen hat, bekommt die Ethik einen immer größeren Stellenwert in der KI. Die Ethik an sich ist die Grundlage einer Gesellschaft, die Ethik ist neben dem Recht eine Leitlinie zum Verhalten in der Gesellschaft. Um die KI und ethischen Wertvorstellungen zu vereinen, werden Leitlinien aus der Wirtschaft und Wissenschaft getrieben. Ein Konzept ist die vertrauenswürdige KI, bestehend aus Recht und Ethik. So sollen wenn KI eingesetzt wird, die folgenden Grundwerte gewährleistet werden: Autonomie/Selbstbestimmtheit des Menschen, fairness, Transparenz, Verlässlichkeit, Sicherheit und Datenschutz. Besonders der Aspekt der Fairness spielt für die Arbeit eine besondere Rolle. Da KI in der Lage ist, abhängig von ihrer Funktionsweise, für unzulässige Unfaire zu sein. Insbesondere für Entscheidungen spielt Fairness eine sehr große Rolle und muss berücksichtigt werden. Diese nicht vorhandene Fairness kommt jedoch nicht von irgendwo her, sondern durch einen Bias.\\
Ein Bias beschreibt im Allgemeinen eine Verzerrung. Im Kontext der KI ist eine Verzerrung oft eine nicht wahrheitsgemäße Repräsentation der Realität, z.B. durch unzulässige Vorurteile. Im Bereich des ML gibt es eine Vielzahl an unterschiedlichen Arten von Bias. Dabei kann ein Bias durch einen Menschen, Daten oder Algorithmen erzeugt sein. Bei Menschen sind dies in der Regel Vorurteile durch die Gesellschaft, das Verhalten oder die Vergangenheit geprägt. Algorithmen hingegen können durch falsch erlernte Zusammenhänge falsche Schlüsse ziehen und so Verzerrungen von Ergebnissen erzeugen.\\
Der entscheidende und für diese Arbeit wichtigste Bias ist jedoch der, der durch die Daten erzeugt wird. Es gibt unterschiedliche Arten von Bias in Daten, wie z.B. die Auswahl der Trainingsdaten (sampling Bias) bis hin zu deren Korrektheit (measurement Bias). Eine häufige Art der Verzerrung ist die der Trainingsdaten, wobei man von Lable Bias spricht. Dabei ist die Beschriftung der Trainingsdaten verzerrt. Dies entsteht, da Trainingsdaten nahezu immer von Menschen "gelabelt" und im superviced learning menschliche Verhaltensweisen reproduziert werden. Wenn nun der Mensch bei der Beschriftung der Trainingsdaten Vorurteilbehaftet gehandelt hat, ist diese Verzerrung in den Daten und somit auch dem Modell vorhanden. Daher ist der Lable Bias eigentlich ein durch den Menschen als Bewertenden geschaffenes Problem. So kam es beispielsweise schon zu einer KI, welche Sexistisch im Einstellungsprozess von Amazon handelte.\\
Um solche Dinge vorzubeugen und das Ziel der Arbeit zu erreichen, wurden zwei möglichst realitätsnahe Szenarien entwickelt, für welche Daten mit einem Bias beinhaltend generiert werden. Damit können die daraus generierten Daten in der Lehre verwendet werden, um ein Bewusstsein für Bias in Daten zu schaffen. Das erste Szenario behandelt die Bewertung von Bewährungsanträgen mit stattgegeben oder nicht. Dabei besteht ein Antrag aus den 5 Attributen Name, Geschlecht, Hautfarbe, Härte des Vergehens und die noch laufende Strafe. Das zweite Szenario behandelt die Bewertung eines sozialen Punktesystems. Dabei werden Personen nach 7 angegebenen Attributen mit einer Punktzahl zwischen 600 und 1400 bewertet.\\
Für beide Szenarien wurde ein Datengenerator entwickelt, welcher ein möglichst realitätsnahes Datenset zurückliefert. Dafür werden im Datengenerator zuerst das Rohdatenset basierend auf Wahrscheinlichkeiten aus aktuellen Statistiken erstellt und anschließend dieses Datenset nach bestimmten Regeln bewertet. Bei der Bewertung wird ein von der benutzenden Person gewünschter Bias hinzugefügt. Der final bewertete Datensatz kann dann in der Datenauswertung mit dem Tool Tableau und unterschiedlichsten Diagrammen auf den Bias untersucht werden. Dabei ist der Bias in einer sogenannten Story im Tableau Stück für Stück aufgezeigt wurden. So kann durch diese beiden Szenarien und durch die passenden Datengeneratoren unterschiedliche Datensets generiert werden, welche dann in der Lehre untersucht werden können. Durch die Untersuchung kann der Bias aufgezeigt werden und dadurch ein Bewusstsein für diesen geschaffen werden.\\\\
Abschließend kann damit gesagt werden, dass der Datengenerator es uns ermöglicht an einem praktischen Beispiel einen Bias in Daten zu ermitteln. Zudem kann das Projekt hinsichtlich des Ziels ein Lehrmittel zu schaffen als vollen Erfolg betrachten. Den man kann Daten generieren, diese individuell verzerren und besitzt eine beliebig große Menge an qualitativ guten Daten für die Verwendung als Trainingsdaten. So kann das Bewusstsein für Bias in Daten in der Lehre gesteigert werden.\\
Reflektierend ist jedoch anzumerken, dass es sich hierbei noch um ein sehr junges Forschungsgebiet handelt. So existieren nur wenige Lösungen einen bekannten Bias zu minimieren oder damit umzugehen. Oft ist hierbei die einzige Lösung neue Daten zu erzeugen oder Einbußen in der Genauigkeit zu bekommen durch das Entfernen von Attributen. Zudem gibt es dabei auch viele technische Herausforderungen. Aber auch organisatorisch müssen Methoden entwickelt werden, wie man in der Lage ist frühzeitig Verzerrungen zu erkennen, Gegenmaßnahmen zu definieren und präventiv zu verhindern. In beiden Bereichen steht man noch vor großen Herausforderungen, da die Auswirkungen eines Bias oft auch unberechenbar und teils nicht abzuschätzen sind. Es kann harmlose Folgen haben, wie einen Skandal, für den man sich entschuldigen muss, aber auch deutlich drastischere Folgen, wenn man sich beispielsweise den Bereich Gesundheitswesen oder Justiz ansieht.\\
Daher muss eine KI zukünftig vertrauenswürdig sein und dabei sowohl die rechtlichen Begebenheiten als auch die Gesellschaftlichen ethischen Wertvorstellungen in ihrem Handeln erfüllen. Deshalb ist ein erster Schritt die Aufklärung über die Existenz von Verzerrungen und Bias.
\newpage
\section{Ausblick}
\label{ausblick}
Die beiden Datengeneratoren für das jeweilige Szenario können optimal von einer Person verwendet werden, um beliebig große Datensets mit einem gewünschten Bias zu generieren. Dabei kann das erste Szenario und die dazugehörige Datenauswertung so sehr einfach für den Einstieg in das Thema verwendet werden, da dieses etwas einfacher gehalten ist und der Bias leicht zu erkennen ist. Das zweite Szenario und die zugehörige Auswertung können dann anschließend verwendet werden, falls mehr Interpretation und Verbindungen in den Daten gewünscht ist. Dabei ist der Bias jedoch auch schwerer zu finden. Für das erste Szenario kann in der Datengenerierung noch das Attribut der Laufenden Strafe realitätsnaher gestaltet werden. Dies ist eine Möglichkeit für die Zukunft, um dies zu optimieren. Zudem können noch mehr Verbindungen in den Daten geschaffen werden. Für das zweite Szenario können ebenfalls zwei Attribute realitätsnaher gestaltet werden. Zudem besteht auch noch Erweiterungspotential in der Datenauswertung, diese kann noch um einige Diagramme erweitert und vertieft werden.\\
So kann auch auf Grundlage der bisherigen zwei Szenarien ein weiteres drittes Szenario entwickelt und umgesetzt werden.\\
Für die Lehre besteht die Möglichkeit neben der Betrachtung der Datenauswertung auch ein Model auf Basis der generierten Daten zu trainieren, um daraus dann den Bias zu entdecken und die Ursachen eines Bias live zu erleben.\\ 
Zudem kann versucht werden operationale und technische Lösungen zu entwickeln einen Bias zu entfernen bzw. zu verhindern. Ein Ansatz dies zu Unterstützen wäre auch ein Konzept für vertrauenswürdige KI einzuführen. Dadurch wäre eine Zertifizierung gegeben und damit eine Sicherheit, dass gewisse Systeme Grundanforderungen erfüllen.\\\\
Dadurch, dass es sich um so einen neuen Forschungsbereich handelt, wurden von dem Thema einige Facetten noch nicht betrachtet. Eine Veröffentlichung, welche hier für den Ausblick nennenswert ist, setzt sich mit den Vorteilen von Bias im ML auseinander. Diese wollen den Bias bewusst in Daten erzeugen, um ein Modell für ethische Richtlinien zu sensibilisieren. Dadurch wollen diese nicht die Symptome, welche durch die Verzerrungen entstehen, bekämpfen sondern die Ursache das Model selbst. Anstatt die Algorithmen zu manipulieren und die Trainingsreize zu unterdrücken, soll der Bias bewusst eingesetzt werden. Das Filtern von Daten ist nur Symptom Bekämpfung und nicht die Ergründung des Problems von Voreingenommenheit, egal ob in Lernalgorithmen oder Menschen. Dies ist ein völlig neuer höchst interessanter Ansatz mit Bias in der KI umzugehen und dadurch möglicherweise die KI sogar zu stärken.\cite{Fabi2022}
\end{onehalfspace}
\newpage