\chapter{Einleitung}
    \label{sec:einleitung}
    Idee:
    Digitalisierung -> Digitalisierung durch KI -> KI durch Daten -> Daten sind Öl der Zukunft -> Was sind Daten
    
    Die fortschreitende Digitalisierung ist kaum noch aus unserem Alltag wegzudenken. Durch immer mehr Programme die einem den Alltag erleichtern sollen, nutzen wir die Errungenschaften der Digitalisierung täglich. Am häufigsten wird dabei von künstlicher Intelligenz geredet und meist ist uns nicht einmal Bewusst, dass im Hintergrund mit künstlicher Intelligenz gearbeitet wird. Egal ob als intelligenten Routenplaner oder Sprachsteuerung, all diese Anwendung basieren auf \ac{KI}


    Alter Text:
    
    Die Digitalisierung ist aus unserem Leben nicht mehr wegzudenken. Viele Kleinigkeiten aus unserem Alltag werden bereits durch die Digitalisierung vereinfacht. Teils ist uns dies nicht einmal Bewusst. Neue Möglichkeiten die nur durch künstliche Intelligenz möglich sind, sind in unserem Alltag überall auffindbar. Egal ob bei der Nutzung von Nagivationssystemen wie Google Maps und einer optimierten Routenplanung, oder 

    Die Digitalisierung ist jedoch nur eine Folge daraus, dass Daten gesammelt und analysiert werden. 
    
    Die Digitalisierung ist aus unserem Leben nicht mehr wegzudenken. Sie beeinflusst nahezu unseren ganzen Alltag, egal ob
    \\
    Alter Text:
    \\
    Das Arbeiten mit Daten wird immer relevanter und so werden auch in den Hochschulen und Universitäten immer mehr Vorlesungen im Bezug auf das Arbeiten mit Daten gehalten. Durch zunehmende Internetnutzung und neue Möglichkeiten Daten zu sammeln und zu speichern, wächst die Datenmenge stetig weiter an. Daten allein sind jedoch relativ wertlos, was sie wertvoll macht, ist die Arbeit mit den Daten. Bei dieser Arbeit handelt es sich in den meisten Fällen um die Entwicklung von Maschinellen Lernmodellen oder das Konzipiere und Trainieren einer \ac{KI}, die durch die gesammelten Daten trainiert werden kann. Durch die Verwendung der Daten als Lerndaten für eine \ac{KI}, können Entscheidungen aber auch Vorhersagen auf Basis der in den Lerndaten enthaltenen zusammenhänge treffen. Da sich letzendlich alles um Zusammenhänge in Daten dreht, ist auch ein wichtiges Themenfeld im Bereich der \ac{KI} die Ethik. Da Fragen gestellt werden wie \dq Inwieweit darf die \ac{KI} entscheidungen aufgrund von Datengrundlagen treffe \dq. \cite{dullien2018}

    \section{Motivation}
    \label{subsec:motivation}

    -	Anonymisierung/Pseudonymisierung bei besonders gro{\ss}en Datensätzen ist schwierig \\
    -	Nachvollziehbarkeit von Bias verzerrten Daten \\
    -	Veranschaulichung von Bias in Daten für die Allgemeinheit, um auf das Problem im Bereich ML aufmerksam zu machen 

    \section{Zielsetzung}
    \label{subsec:zielsetzung}
    -	Datengenerator für Bias verzerrte Daten \\
    -	Visualisierung von Bias in Lerndaten für ML \\
    -	Gesamt Produkt zur Erstellung von Daten und derer Bias Visualisierung für die Lehre \\
    \\
    -   1 Satz, was sollen wir machen --> Stichwortliste mit Anforderungen \\
    \\
    -   Maschinelles Lernen hängt von den Trainingsdaten ab.\\
    -   Trainingsdaten können einen Bias Data enthalten.\\

    \section{Aufbau der Arbeit}
    \label{subsec:aufbau der arbeit}


    \newpage