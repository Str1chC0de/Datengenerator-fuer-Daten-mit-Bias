\chapter{Einleitung}
\begin{onehalfspace}    
    \label{sec:einleitung}
        Die fortschreitende Digitalisierung ist kaum noch aus unserem Alltag wegzudenken. Durch immer mehr Programme, die den Alltag erleichtern sollen, nutzen wir die Errungenschaften der Digitalisierung täglich. Häufig ist hier die Rede von künstlicher Intelligenz. Dabei ist uns meist nicht einmal Bewusst, dass im Hintergrund mit künstlicher Intelligenz gearbeitet wird. Egal ob als intelligenten Routenplaner oder Sprachsteuerung, hinter all diese Anwendung steckt heute nicht mehr nur ein Optimierungsalgorithmus sondern \ac{KI}. 
        \\
        Mit der Digitalisierung hat man begonnen große Datenmengen zu sammeln. Durch den technischen Fortschritt im Bereich von Big Data, werden diese Datenmengen heutzutage unvorstellbar groß. Mit dem Erfassen und Speichern von Daten ist man in der Lage seine Produkte stetig zu verbessern und zudem neue Geschäftsmodelle zu schaffen. Zu diesen neuen Geschäftsmodellen gehört die nicht mehr aus unserem Alltag wegzudenkende \ac{KI}. Sie ist in der Lage Entscheidungen und Vorhersagen auf Basis von Daten zu treffen, die durch einen Menschen nur  mit großem Aufwand getätigt werden können.  Egal ob eine Entscheidung oder eine Vorhersage von einer \ac{KI} getroffen wird, sie basiert auf Daten der Vergangenheit. Aus diesem Grund sind Daten, sobald sie verarbeitet und genutzt werden, eine so wertvolle Ressource.
        \\
        Für eine \ac*{KI} werden Daten zum Lernen genutzt. Entscheidend für die Qualität der \ac*{KI} ist somit die Datengrundlage auf der die \ac*{KI} basiert. Lernen bedeutet, dass Zusammenhänge und die dadurch abgebildeten Verhaltensweisen in den Daten von der \ac*{KI} erkannt und gelernt werden. Durch diese Art des Lernens, wie auch wir Menschen lernen, ergeben sich jedoch nicht nur Potentiale sondern auch Risiken. Abhängig von der Datenqualität und Richtigkeit bzw. Zuverlässigkeit der Daten werden zukünftige Entscheidungen und Vorhersagen getroffen. Eine \ac*{KI} betrachtet dabei die Daten vollkommen neutral ohne Hintergrundwissen und ethische Wertvorstellungen. Für manche Entscheidungen gibt es jedoch nicht zwingend Richtig oder Falsch. Häufig ist es ein schmaler Grad dazwischen. In diesen Fällen wird das menschliche Handeln durch Ethik gesteuert. Eine \ac*{KI} besitzt jedoch keine Ethik und so können Entscheidungen einer KI durch unterschiedliche Ursachen benachteiligend oder gar diskriminierenden sein.
        \\
        Durch \ac*{KI} öffnen sich viele neue Möglichkeiten und Geschäftsmodelle. Sie wird in immer mehr Bereichen eingesetzt. Doch wenn eine \ac*{KI} vor moralischen Entscheidungen steht sollte man bedenken, dass eine Maschine keine Ethik und Moral besitzt. Dies kann zu fatalen Fehlentscheidungen führen und \glqq{}the dark side of \ac*{KI}\grqq{} zum vorschein bringen.

    \newpage
    \section{Motivation}
    \label{subsec:motivation}
        Mit den Vorteilen der \ac*{KI} kommen immer auch Nachteile. Um die Schattenseite einer \ac*{KI} verstehen zu können, muss man das Thema \ac*{KI} etwas genauer betrachten. Eine \ac*{KI} ist meist ein Instrument zur Vorhersage oder Erkennung. Die Entscheidungen werden durch maschinelles Lernen getroffen. Beim Maschinellen lernen werden, vereinfacht gesagt, Verhaltensweisen und Zusammenhänge in Daten analysiert und diese für zukünftige Entscheidungen als Vorlage genutzt. Die besondere Eigenschaft hierbei ist, dass die Daten, auch Trainingsdaten genannt, Daten aus der Vergangenheit sind. Das Lernen funktioniert ähnlich wie bei uns Menschen, die \ac*{KI} bekommt Trainingsdaten die zeigen, wie Sie zu Entscheiden hat und übernimmt diese Verhaltensweise. Da eine \ac*{KI} auf diese Art und weiße lernt und Entscheidungen trifft, ist naheliegend, dass es wie beim Menschen durch diese Form des Lernens auch ungewünschte Effekte gibt. Bei uns Menschen lernen wir in der Regel von den Eltern, die einen erziehen. Bei einer \ac*{KI} sind die Eltern die Daten, die Verhaltensweisen beibringen. 
        \\
        Bei der \ac*{KI} und speziell dem \ac{ML} ergeben sich mehrere zu berücksichtigende Probleme. Das häufigste Problem des \ac*{ML} ist das Under- und Overfitting. Dabei wird entweder zu wenig aus den Trainingsdaten gelernt und deshalb willkürlich entschieden oder die Trainingsdaten werden \glqq{}auswendig\grqq{} gelernt und deshalb bei neuen Daten willkürlich entschieden. 
        \\
        Ein unbekannteres Problem von \ac*{KI} und \ac*{ML} ist die Verzerrung in den Trainingsdaten. Wenn Trainingsdaten aufgrund unterschiedlichster Ursachen unerwünschte Zusammenhänge beinhalten, wird von Bias gesprochen. So können zum Beispiel Entscheidungen aufgrund eines unbekannten Zusammenhang in den Trainingsdaten, häufig auf diskriminierenden Verhaltensmustern, basieren. Die Problematik liegt darin, dass den Endnutzer in der Regel nicht bekannt ist, dass es einen Bias in den Daten geben kann. In den meisten Fällen ist eine solche Verzerrung verborgen und wird erst im produktiven Betrieb der \ac*{KI} festgestellt.
        \\
        Diese Verzerrungen führen meist zu Skandalen in der Medienwelt. Es wurde bereits diverse Male in der Presse darüber berichtet, dass bspw. in Unternehmen Bewerbungen durch ein \ac*{KI} vorsortiert wurden und dabei Personen mit Migrationshintergrund aus nicht nachvollziehbaren Gründen aussortiert wurden. Ein solches diskriminierendes Verhaltensmuster wurde daraufhin in den Trainingsdaten erkannt.
        \\
        Diese Diskriminierungen sind jedoch nicht zu vergessen immer auf Trainingsdaten und so in der Regel auf reale Daten aus der Vergangenheit zurückzuführen. Das Problem des Bias in Daten ist daher, durch menschliches Verschulden, eine Schattenseite der \ac*{KI} 

    \newpage
    \section{Zielsetzung}
    \label{subsec:zielsetzung}
    \ac*{KI} ist in allen Lebensbereichen vorhanden und auch nicht mehr wegzudenken. Jedoch die Schattenseite der \ac*{KI}, ist den meisten Menschen unbekannt. Dabei spielt die Ethik eine besondere Rolle, denn im Gegensatz zu uns Menschen, verfügt eine \ac*{KI} nicht über ethische Werte und Moral. Häufig spielt die Ethik jedoch in der Entscheidungsfindung eine nicht zu vernachlässigende Rolle. Die Folge aus der fehlenden Ethik bei einer \ac*{KI} kann zu Fehlentscheidungen und fatalen Folgen führen.
    \\
    Aus diesem Grund soll mehr Bewusstsein für Bias in Daten geschaffen werden. Insbesondere die Entwickler von \ac*{KI} Lösungen müssen für die Thematik mehr sensibilisiert werden, sodass mögliche Benachteiligungen nicht erst in der Praxis festgestellt werden. Dafür soll ein Datengenerator, welcher Daten mit Bias erzeugt entwickelt werden. Um diese Daten in der Lehre einsetzten zu können soll zusätzlich eine Auswertung entwickelt werden, welche den Bias als Visualisierung veranschaulicht. 
    \\
    Die Umsetzung liegt den folgenden Anforderungenzugrunde:
    \begin{itemize}
        \item Konzeption zweier Szenarien, die realitätsnah sind.
        \item Erstellung eines Datengenerators für zufalls generierte Daten. 
        \begin{itemize}
            \item Python Script zum generieren eines großen Datensets
            \item Erzeugung eines Bias durch die Bewertung des Datensets
            \item Bewertete Daten als CSV Datei bereitstellen
        \end{itemize}
        \item Erstellung einer Auswertung zur Veranschaulichung des Bias.
        \begin{itemize}
            \item Visuelle Auswertung in Tableau
        \end{itemize}
    \end{itemize} 
    Ziel ist es, einen Datengenerator für Daten mit Bias zu entwickeln und zusätzlich eine visualisierte Auswertung, die den Bias veranschaulicht. Dieser soll in der Lehre zum Einsatz kommen und für die Thematik von Bias in Daten sensibilisieren.

    \newpage
    \section{Aufbau der Arbeit}
    \label{subsec:aufbau der arbeit}
        Der erste Abschnitt ist in drei Passagen aufgeteilt. Zu Beginn wird das allgemeine Thema der Daten als Grundlage für \ac*{KI} betrachtet. Dabei wird insbesondere auf die Datenqualität eingegangen. Des weiteren wird das Thema Bias, also die Verzerrung in den Daten, auf Basis der Literatur veranschaulicht. In der folgenden Passage wird auf \ac*{KI} und \ac*{ML} eingegangen. Ebenso wird die Ethik in der \ac*{KI} betrachtet. Die letzte Passage setzt sich dann mit Bias in \ac*{KI} Trainingsdaten auseinander. Dabei liegt der Fokus auf der möglicherweise entstehenden Diskriminierung. Im Gegensatz dazu werden zusätzlich Ansätze und Konzepte von Gegenmaßnahmen betrachtet.
        Im nächsten großen Abschnitt wird die praktische Umsetzung des Datengenerators näher betrachtet. Dafür werden zu Beginn die zwei Szenarien ausgearbeitet und näher beschrieben. Als nächstes werden die daraus entstehenden Anforderungen in Form eines Konzepts aufgestellt. Dieses unterscheidet sich in Fein und Grobkonzept und beschreibt die logischen Funktionen. Daraufhin folgt die Implementierung des beschriebenen Konzepts. Anschließend folgt die in den Anforderungen geforderte Auswertung der generierten Daten. Dazu wird die erstellte Auswertung in Tableau herangezogen. Zum Schluss dieses Abschnitts wird das Ergebnis des Datengenerators und der Auswertung vorgestellt und evaluiert.
        Abschließend werden alle Erkenntnisse gesammelt und zusammengefasst. Hier wird auch das Ergebnis der Arbeit kritisch Reflektiert und evaluiert. Beendet wird die Arbeit mit einem Ausblick darüber, welche Relevanz Bias in der \ac*{KI} zukünftig haben könnte. 
    
    \newpage
\end{onehalfspace}