\chapter{Stand der Technik}
    \begin{onehalfspace}  
        \label{sec:theorie/standdertechnik}
            In diesem Kapitel wird der Stand der Technik näher beleuchtet. Der Fokus liegt dabei auf den Themen Daten, \ac*{KI} und Bias. Zu Beginn wird auf basis der Literatur erläutert, was Daten sind, was Datenqualität bedeutet und worum es sich bei einem Bias handelt. Daraufhin wird näher auf \ac*{KI}, das Teilgebiet \ac*{ML} und die Ethik in der \ac*{KI} eingegangen. Zuletzt werden die Themen in einen gemeinsamgen Kontext gesetzt und der Einfluss eines Bias auf eine \ac*{KI} betrachtet und Gegenmaßnahmen untersucht. 
        
        \section{Daten}
        \label{subsec:daten}
        
        \subsection{Datenqualität}
        \label{subsubsec:datenqualität}

        \subsection{Bias}
        \label{subsubsec:Bias}
            -   Begriffserklärung: Data Bias vs Bias Verzerrung (zu viel/zu wenig lernen im ml)\\
            -   Arten von Bias: \\
                -   Bias durch Abwesenheit - Wenn eine Info fehlt, kann das zu Diskriminierung führen. \\
                -   Diskriminierung durch Menschen. \\
            Arten von Bias: Cognitive, Social, Perceptual und Motivational Bias \cite{Bias}
 
        \section{Künstliche Intelligenz}
        \label{subsec:KIandML}

        \subsection{Was ist Künstliche Intelligenz}
        \label{subsec:wasistKI}

        \subsection{Teilgebiet Maschinelles Lernen}
        \label{subsubsec:teilgebietML}
            -   Superviced learning \\
            -   Unsuperviced learning \\

        \subsection{Ethik in der Künstlichen Intelligenz}
        \label{subsubsec:ethikinderKI}

        \section{Bias in der \ac*{KI}}
        \label{subsec:KIundbias}

        \subsection{Diskriminierung durch verzerrte Daten}
        \label{subsubsec:diskriminierungdurchverzerrung}

        \subsection{Gegenma{\ss}nahmen}
        \label{subsubsec:gegenmassnahmen}
            -   Wenn der Parameter mit dem Bias entfernt wird, wird das Ergebnis erstmal schlechter. 
            \\

    \newpage
    \end{onehalfspace}